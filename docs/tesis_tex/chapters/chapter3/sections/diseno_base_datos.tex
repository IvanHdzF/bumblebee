\label{subsec:db_guard}

\subsection{Base de datos de seguridad}
Esta base de datos concentra la información necesaria para gestionar el acceso al sistema. Contiene el catálogo de roles disponibles, los usuarios registrados y los tokens de refresco de sesión. Bajo este esquema se asegura que cada usuario esté vinculado a un rol específico y que cada token de sesión pueda asociarse a un usuario.
\paragraph{Entidades y relaciones}
\begin{itemize}
  \item \textbf{user\_role}: catálogo de roles.
  \item \textbf{app\_user}: usuarios de la aplicación; cada usuario pertenece a un rol.
  \item \textbf{refresh\_token}: token asociado a un usuario y con vencimiento.
\end{itemize}

\paragraph{Esquema lógico (PostgreSQL)}
% Los nombres se muestran en snake_case; ajustar a la estrategia de nombres de JPA/Hibernate si difiere.
\begin{table}[H]
\centering
\caption{Tablas y llaves en la base de datos \texttt{beehive\_guard}}
\label{tab:guard_schema}
\begin{tabular}{|l|l|l|}
\hline
\textbf{Tabla} & \textbf{Llave primaria} & \textbf{Llave foranea} \\ \hline
user\_role & id BIGSERIAL & -- \\ \hline
app\_user & id UUID & role\_id \(\rightarrow\) user\_role(id) \\ \hline
refresh\_token & id BIGSERIAL &
app\_user\_id \(\rightarrow\) app\_user(id) \\ \hline
\end{tabular}
\end{table}

\paragraph{DDL de referencia}

\begin{lstlisting}[language=SQL, label={lst:guard-database-schema}, caption={Esquema de la base de datos guard-database}]
-- PostgreSQL 14+
CREATE EXTENSION IF NOT EXISTS pgcrypto; -- gen_random_uuid()

CREATE TABLE user_role (
  id BIGSERIAL PRIMARY KEY,
  description TEXT NOT NULL
);

CREATE TABLE app_user (
  id UUID PRIMARY KEY DEFAULT gen_random_uuid(),
  name TEXT,
  phone_number TEXT,
  email_address TEXT NOT NULL UNIQUE,
  role_id BIGINT NOT NULL REFERENCES user_role(id),
  password TEXT NOT NULL
);

CREATE TABLE refresh_token (
  id BIGSERIAL PRIMARY KEY,
  token TEXT NOT NULL UNIQUE,
  app_user_id UUID NOT NULL
    REFERENCES app_user(id) ON DELETE CASCADE,
  expiry_date TIMESTAMPTZ NOT NULL
);

\end{lstlisting}

\subsection{Base de datos de colmenares}
Esta base de datos almacena la información relativa a colmenares (\texttt{apiary}), colmenas (\texttt{beehive}) y las mediciones asociadas a cada colmena (\texttt{measure}).
El esquema garantiza la relación jerárquica entre apiarios y colmenas, y la persistencia histórica de las mediciones.

\paragraph{Entidades y relaciones}
\begin{itemize}
    \item \textbf{apiary}: catálogo de apiarios, identificados por su propietario y nombre.
    \item \textbf{beehive}: colmenas asociadas a un apiario, con identificación única mediante número de serie.
    \item \textbf{measure}: registros de temperatura, humedad, peso y audio de una colmena, junto con su clasificación como anómala o normal
\end{itemize}

\paragraph{Esquema lógico (PostgreSQL)}
\begin{table}[H]
\centering
\caption{Tablas y llaves en la base de datos \texttt{beehive\_nest}}
\label{tab:apiary_schema}
\begin{tabular}{|l|l|l|}
\hline
\textbf{Tabla} & \textbf{Llave primaria} & \textbf{Llave foránea} \\ \hline
apiary & id BIGSERIAL & -- \\ \hline
beehive & id BIGSERIAL & apiary\_id \(\rightarrow\) apiary(id) \\ \hline
measure & id BIGSERIAL & beehive\_id \(\rightarrow\) beehive(id) \\ \hline
\end{tabular}
\end{table}

\paragraph{DDL de referencia}

\begin{lstlisting}[language=SQL, label={lst:apiary-database-schema}, caption={Esquema de la base de datos beehive-nest}]
-- PostgreSQL 14+

CREATE TABLE apiary (
  id BIGSERIAL PRIMARY KEY,
  name TEXT NOT NULL,
  owner UUID NOT NULL
);

CREATE TABLE beehive (
  id BIGSERIAL PRIMARY KEY,
  name TEXT,
  serial TEXT NOT NULL UNIQUE,
  latitude DOUBLE PRECISION,
  longitude DOUBLE PRECISION,
  apiary_id BIGINT NOT NULL REFERENCES apiary(id) ON DELETE CASCADE
);

CREATE TABLE measure (
  id BIGSERIAL PRIMARY KEY,
  time TIMESTAMPTZ NOT NULL,
  beehive_id BIGINT NOT NULL REFERENCES beehive(id) ON DELETE CASCADE,
  temperature DOUBLE PRECISION,
  humidity DOUBLE PRECISION,
  weight DOUBLE PRECISION,
  audio_recording_url TEXT,
  label TEXT
);
\end{lstlisting}

\subsection{Base de datos de características}
La base de datos \texttt{mind-database} almacena las características acústicas extraídas de cada medición. Cada registro en la tabla \texttt{features} se vincula a una medición específica. Esta base de datos solo contiene una tabla.

\paragraph{Entidades y relaciones}
\begin{itemize}
  \item \textbf{features}: almacena las características calculadas a partir de una medición de audio. Se relaciona con la tabla \texttt{measure} mediante la llave foránea \texttt{measurement\_id}.
\end{itemize}

\paragraph{Esquema lógico (PostgreSQL)}
\begin{table}[H]
\centering
\caption{Tabla unica en la base de datos \texttt{mind-database}}
\label{tab:features_schema}
\begin{tabular}{|l|l|l|}
\hline
\textbf{Tabla} & \textbf{Llave primaria} & \textbf{Llave foránea} \\ \hline
features & id BIGSERIAL & measurement\_id \(\rightarrow\) measure(id) \\ \hline
\end{tabular}
\end{table}

\paragraph{DDL de referencia}

\begin{lstlisting}[language=SQL, label={lst:features-database-schema}, caption={Esquema de la tabla features en mind-database}]
-- PostgreSQL 14+

CREATE TABLE features (
  id BIGSERIAL PRIMARY KEY,
  measurement_id UUID NOT NULL REFERENCES measure(id) ON DELETE CASCADE,
  date DATE NOT NULL,
  zero_crossing_rate DOUBLE PRECISION,
  energy DOUBLE PRECISION,
  energy_entropy DOUBLE PRECISION,
  spectral_centroid DOUBLE PRECISION,
  spectral_spread DOUBLE PRECISION,
  spectral_entropy DOUBLE PRECISION,
  spectral_flux DOUBLE PRECISION,
  spectral_rolloff DOUBLE PRECISION,
  mfcc_1 DOUBLE PRECISION,
  mfcc_2 DOUBLE PRECISION,
  mfcc_3 DOUBLE PRECISION,
  mfcc_4 DOUBLE PRECISION,
  mfcc_5 DOUBLE PRECISION,
  mfcc_6 DOUBLE PRECISION,
  mfcc_7 DOUBLE PRECISION,
  mfcc_8 DOUBLE PRECISION,
  mfcc_9 DOUBLE PRECISION,
  mfcc_10 DOUBLE PRECISION,
  mfcc_11 DOUBLE PRECISION,
  mfcc_12 DOUBLE PRECISION,
  mfcc_13 DOUBLE PRECISION,
  label VARCHAR(50)
);
\end{lstlisting}
