
% Evidenciar problematica existente y porque es deseable que haya abejas en la region
% Describir la situacion actual de Global -> Mexico -> Regional
% Hablar de las soluciones actuales
% Parafrasear objetivo
% Metodologia a seguir
% Resultados esperados
% Explicar propuesta de solucion

% Hacer 1 pagina y 1/2


La apicultura es la actividad que se encarga de la crianza y explotación de abejas, de la cual dependen miles de apicultores a nivel mundial, y más de 14 mil apicultores a nivel nacional \cite{data_mexico_2023a}.
Las colmenas apícolas tienen 2 funciones, la primera es la elaboración de productos provenientes de las abejas, entre estos se encuentran el polen, los propóleos, la cera, la jalea real y el principal producto, la miel. Además, las colmenas brindan un servicio de polinización a las plantas con flor que se encuentren a los alrededores de la colmena \cite{bradbear_2005}.
Como ya se mencionó, una de las funciones de las colmenas es la polinización. Este proceso es fundamental para la reproducción de plantas con flor, y se realiza al momento en el que una abeja lleva polen de la parte masculina de una planta a una parte femenina \cite{bradbear_2005}.

\textit{El texto de la introducción está bien, pero te recomiendo que en otro párrafo hables sobre cómo se está utilizando la IA en la actualidad, para mejorar sobre todo la polinización ya sea con alertas a través del monitoreo de la actividad al interior de la colmena con IoT, o con la utilización de algoritmos de IA..., finalmente, puedes describir brevemente 1 o 2 de los trabajos relacionados, más interesantes o más parecidos al tuyo. En otro párrafo, debes mencionar el objetivo que se pretende lograr con tu proyecto final, y redáctalo de tal forma que no pongas el objetivo general, sino presentarlo como propuesta de solución. No olvides incluir las referencias.}
