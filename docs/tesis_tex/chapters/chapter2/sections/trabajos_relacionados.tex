La adición de tecnologías de internet de las cosas (IoT) en la apicultura ha significado un cambio revolucionario a comparación de las prácticas tradicionales, aumentando la eficiencia, productividad y escalabilidad. Estas tecnologías involucran el uso de diferentes sistemas interconectados, incluyendo sensores, microcontroladores y software, con el objetivo de ayudar a monitorear y administrar las actividades apícolas en tiempo real.
En este capítulo se exploran las aplicaciones relevantes en las cuales se han desarrollado las aplicaciones de IoT en el monitoreo de colmenas apícolas, con el objetivo de dar seguimiento a la salud y actividad de las colonias de abejas, proveyendo así información de la colmena, incluyendo parámetros como temperatura, humedad, peso y acústica que ayudan a los apicultores a mantener colonias saludables y tomar decisiones acordes a la información en tiempo real.
La arquitectura de sistemas IoT para monitoreo de colmenas involucra varios componentes clave, incluyendo nodos de sensores, sistemas de transmisión de datos, sistemas de energía y el componente de procesamiento de datos. En este capítulo se examinan las prácticas utilizadas para implementar estos sistemas.

\subsection{Embedded Software for IoT Bee Hive Monitoring Node}
Este artículo presenta un sistema embebido para monitoreo de variables relevantes; se propone un sistema que tiene las capacidades de monitorear humedad, temperatura, peso y audio. Sin embargo, solo se detalla la implementación del monitoreo de humedad y temperatura. Para este sistema se utilizó el sensor SHT21~\cite{sht21} para la medición de temperatura y humedad.~\cite{vidrascu_svasta_2017a}  
El sistema está construido alrededor del componente ESP8266 que integra un microcontrolador Tensilica L106 de 32 bits en conjunto con el módulo transmisor Wi-Fi que integra circuitos de radiofrecuencia y es capaz de transmitir datos mediante interfaces digitales como SPI, I2C, y UART. Además, el módulo ESP8266 cuenta con tres modos de operación: activo, dormido y dormido profundo, los cuales son aprovechados por el sistema para optimizar el uso de energía.~\cite{vidrascu_svasta_2017a}

\subsection{A Smart Sensor-Based Measurement System for Advanced Bee Hive Monitoring}
Este sistema desarrolla un monitoreo de parámetros relevantes para la colmena, incluyendo peso, sonido, temperatura, humedad y CO$_2$.~\cite{cecchi_spinsante_terenzi_orcioni_2020}  
Es modular y se compone de dos módulos principales:  
\begin{itemize}
    \item \textbf{Módulo Abeja}: Raspberry Pi 3B~\cite{buy_raspberry_pi3_model_b}, tarjeta de sonido UCA22~\cite{behringer_uca222}, micrófonos ADMP401 MEMS~\cite{admp401_datasheet}, sensores DHT22~\cite{liu}, células de carga TAL220~\cite{loadcell_tal220_sparkfun} con HX711~\cite{hx711_sparkfun} y sensor de CO$_2$ TL6615~\cite{t6615_telaire}.
    \item \textbf{Módulo Reina}: Raspberry Pi 3B, sensores DHT22 y un puente 5G (no especificado) utilizado para comunicarse con el servidor remoto.~\cite{cecchi_spinsante_terenzi_orcioni_2020}
\end{itemize}  
Este artículo se energiza mediante una conexión a la red eléctrica, los módulos Reina y Abeja consumen 4 y 4.2 watts respectivamente.~\cite{cecchi_spinsante_terenzi_orcioni_2020}

\subsection{High Reliability Wireless Sensor Node for Bee Hive Monitoring}
Este sistema está diseñado para condiciones de entre -20 a 60 °C con una arquitectura modular que permite intercambiar sensores mediante interfaz I2C.~\cite{vidrascu_svasta_vladescu_2016}  
Incluye sensores SHT21, sistema HX711 con células de carga, sensor de inclinación DMA08 y micrófono MP34DT05-A.~\cite{vidrascu_svasta_vladescu_2016}  
Además, utiliza un módulo ESP8266 serie 32-bit a 80MHz para conexión Wi-Fi directa.~\cite{vidrascu_svasta_vladescu_2016}  
En cuanto a la energía, se emplean supercapacitores XV de Eaton~\cite{supercapacitor_xv_eaton} junto con paneles solares (no especificados) y un convertidor step up.~\cite{vidrascu_svasta_vladescu_2016}

\subsection{Bee Swarm Activity Acoustic Classification for an IoT-Based Farm Service}
Este sistema utiliza un micrófono TDK InvenSense ICS-40300 y una placa Atmel ATmega32U4 sobre la colmena, procesando los datos en un servidor remoto.~\cite{zgank_2019}  
El procesamiento de datos acústicos tiene como objetivo la detección de enjambres, utilizando datos del Open Source Beehives Project (OSBP)~\cite{open_source_beehives_project_iaac}. Se aplican métodos MFCC y LPC para extracción de características y HMM/GMM para clasificación.~\cite{zgank_2019}

\subsection{Maintenance-free IoT Gateway Design for Bee Hive Monitoring}
Este módulo recopila información con sensores SHT21, MPL3115A2~\cite{nxp_mpl3115a2}, TSL2561~\cite{tsl2561_adafruit}, HX711, VEML6075~\cite{adafruit_veml6075} y GP2Y1010AU0F~\cite{vidrascu_svasta_2017b}.~\cite{vidrascu_svasta_2017b}  
Para transmisión de datos se utiliza un ESP8266-12 (A.I.Thinker) y un módem A6 GSM/GPRS (no en bib).~\cite{vidrascu_svasta_2017b}  
El sistema obtiene energía de paneles solares (no especificados) con reguladores LT3652~\cite{lt3652_datasheet} y LTC3652~\cite{ltc3652_datasheet}.~\cite{vidrascu_svasta_2017b}

\subsection{A Pi-Based IoT System Design}
Este sistema integra la suite de sensores GrovePi: humedad, temperatura, GPS, sonido e imágenes térmicas.~\cite{chen_chien_hsu_jing_lin_lin_2020}  
Implementa múltiples módulos Wi-Fi conectados a un módem independiente con conexión a red móvil.~\cite{chen_chien_hsu_jing_lin_lin_2020}  
Se alimenta con una batería de 10000 mAh (180000 J a 5V), con autonomía de 27–33 h.~\cite{chen_chien_hsu_jing_lin_lin_2020}

\subsection{Conclusión}
A lo largo de este capítulo, se han explorado diversas arquitecturas y tecnologías utilizadas en estos sistemas, destacando los componentes esenciales como los nodos de sensores y los sistemas de transmisión de datos, así como las soluciones innovadoras para el manejo de energía y el procesamiento de datos.
\begin{itemize}
    \item \textbf{Nodos de Sensores}: Los sistemas presentados varían en complejidad, pero todos incluyen sensores de temperatura, humedad, peso y acústica.
    \item \textbf{Sistemas de Transmisión de Datos}: Predomina el uso del módulo ESP8266, con casos que incorporan GSM/GPRS.
    \item \textbf{Manejo de Energía}: Paneles solares, baterías y supercapacitores son las soluciones más comunes.
    \item \textbf{Procesamiento de Datos}: Se emplean MFCC, LPC, HMM y GMM en clasificación de eventos acústicos.
\end{itemize}
La adición de tecnologías de internet de las cosas (IoT) en la apicultura ha significado un cambio revolucionario a comparación de las prácticas tradicionales, aumentando la eficiencia, productividad y escalabilidad. Estas tecnologías involucran el uso de diferentes sistemas interconectados, incluyendo sensores, microcontroladores y software, con el objetivo de ayudar a monitorear y administrar las actividades apícolas en tiempo real.
En este capítulo se exploran las aplicaciones relevantes en las cuales se han desarrollado las aplicaciones de IoT en el monitoreo de colmenas apícolas, con el objetivo de dar seguimiento a la salud y actividad de las colonias de abejas, proveyendo así información de la colmena, incluyendo parámetros como temperatura, humedad, peso y acústica que ayudan a los apicultores a mantener colonias saludables y tomar decisiones acordes a la información en tiempo real.
La arquitectura de sistemas IoT para monitoreo de colmenas involucra varios componentes clave, incluyendo nodos de sensores, sistemas de transmisión de datos, sistemas de energía y el componente de procesamiento de datos. En este capítulo se examinan las prácticas utilizadas para implementar estos sistemas.

\subsection{Embedded Software for IoT Bee Hive Monitoring Node}
Este artículo presenta un sistema embebido para monitoreo de variables relevantes; se propone un sistema que tiene las capacidades de monitorear humedad, temperatura, peso y audio. Sin embargo, no se implementó el monitoreo de peso y audio. Para este sistema se utilizó el sensor SHT21~\cite{sht21} para la medición de temperatura y humedad.~\cite{vidrascu_svasta_2017a}  
El sistema está construido alrededor del componente ESP8266 que integra un microcontrolador Tensilica L106 32-bit en conjunto con el módulo transmisor Wi-Fi que integra circuitos de radio frecuencia y es capaz de transmitir datos mediante interfaces digitales como SPI, I2C, y UART. Además, el módulo ESP8266 cuenta con tres modos de operación: activo, dormido y dormido profundo, los cuales son aprovechados por el sistema para optimizar el uso de energía.~\cite{vidrascu_svasta_2017a}

\subsection{A Smart Sensor-Based Measurement System for Advanced Bee Hive Monitoring}
Este sistema desarrolla un monitoreo de parámetros relevantes para la colmena, incluyendo peso, sonido, temperatura, humedad y CO$_2$.~\cite{cecchi_spinsante_terenzi_orcioni_2020}  
Es modular y se compone de dos módulos principales:  
\begin{itemize}
    \item \textbf{Módulo Abeja}: Raspberry Pi 3B~\cite{buy_raspberry_pi3_model_b}, tarjeta de sonido UCA22~\cite{behringer_uca222}, micrófonos ADMP401 MEMS~\cite{admp401_datasheet}, sensores DHT22~\cite{liu}, células de carga TAL220~\cite{loadcell_tal220_sparkfun} con HX711~\cite{hx711_sparkfun} y sensor de CO$_2$ TL6615~\cite{t6615_telaire}.
    \item \textbf{Módulo Reina}: Raspberry Pi 3B, sensores DHT22 y un puente 5G (no especificado) utilizado para comunicarse con el servidor remoto.~\cite{cecchi_spinsante_terenzi_orcioni_2020}
\end{itemize}  
Este artículo se energiza mediante una conexión a la red eléctrica, los módulos Reina y Abeja consumen 4 y 4.2 watts respectivamente.~\cite{cecchi_spinsante_terenzi_orcioni_2020}

\subsection{High Reliability Wireless Sensor Node for Bee Hive Monitoring}
Este sistema está diseñado para condiciones de entre -20 a 60 °C con una arquitectura modular que permite intercambiar sensores mediante interfaz I2C.~\cite{vidrascu_svasta_vladescu_2016}  
Incluye sensores SHT21, sistema HX711 con células de carga, sensor de inclinación DMA08 y micrófono MP34DT05-A.~\cite{vidrascu_svasta_vladescu_2016}  
Además, utiliza un módulo ESP8266 serie 32-bit a 80MHz para conexión Wi-Fi directa.~\cite{vidrascu_svasta_vladescu_2016}  
En cuanto a la energía, se emplean supercapacitores XV de Eaton~\cite{supercapacitor_xv_eaton} junto con paneles solares (no especificados) y un convertidor step up.~\cite{vidrascu_svasta_vladescu_2016}

\subsection{Bee Swarm Activity Acoustic Classification for an IoT-Based Farm Service}
Este sistema utiliza un micrófono TDK InvenSense ICS-40300 y una placa Atmel ATmega32U4 sobre la colmena, procesando los datos en un servidor remoto.~\cite{zgank_2019}  
El procesamiento de datos acústicos tiene como objetivo la detección de enjambres, utilizando datos del Open Source Beehives Project (OSBP)~\cite{open_source_beehives_project_iaac}. Se aplican métodos MFCC y LPC para extracción de características y HMM/GMM para clasificación.~\cite{zgank_2019}

\subsection{Maintenance-free IoT Gateway Design for Bee Hive Monitoring}
Este módulo recopila información con sensores SHT21, MPL3115A2~\cite{nxp_mpl3115a2}, TSL2561~\cite{tsl2561_adafruit}, HX711, VEML6075~\cite{adafruit_veml6075} y GP2Y1010AU0F~\cite{vidrascu_svasta_2017b}.~\cite{vidrascu_svasta_2017b}  
Para transmisión de datos se utiliza un ESP8266-12 (A.I.Thinker) y un módem A6 GSM/GPRS (no en bib).~\cite{vidrascu_svasta_2017b}  
El sistema obtiene energía de paneles solares (no especificados) con reguladores LT3652~\cite{lt3652_datasheet} y LTC3652~\cite{ltc3652_datasheet}.~\cite{vidrascu_svasta_2017b}

\subsection{A Pi-Based IoT System Design}
Este sistema integra la suite de sensores GrovePi: humedad, temperatura, GPS, sonido e imágenes térmicas.~\cite{chen_chien_hsu_jing_lin_lin_2020}  
Implementa múltiples módulos Wi-Fi conectados a un módem independiente con conexión a red móvil.~\cite{chen_chien_hsu_jing_lin_lin_2020}  
Se alimenta con una batería de 10000 mAh (180000 J a 5V), con autonomía de 27–33 h.~\cite{chen_chien_hsu_jing_lin_lin_2020}

\subsection{Conclusión}
A lo largo de este capítulo, se han explorado diversas arquitecturas y tecnologías utilizadas en estos sistemas, destacando los componentes esenciales como los nodos de sensores y los sistemas de transmisión de datos, así como las soluciones innovadoras para el manejo de energía y el procesamiento de datos.
\begin{itemize}
    \item \textbf{Nodos de Sensores}: Los sistemas presentados varían en complejidad, pero todos incluyen sensores de temperatura, humedad, peso y acústica.
    \item \textbf{Sistemas de Transmisión de Datos}: Predomina el uso del módulo ESP8266, con casos que incorporan GSM/GPRS.
    \item \textbf{Manejo de Energía}: Paneles solares, baterías y supercapacitores son las soluciones más comunes.
    \item \textbf{Procesamiento de Datos}: Se emplean MFCC, LPC, HMM y GMM en clasificación de eventos acústicos.
\end{itemize}
