En cuanto al panorama nacional y según la Secretaría de Agricultura y Desarrollo Rural (SADER), México se posicionó como el noveno productor de miel a nivel mundial \cite{SecretariaDeAgriculturaDesarrolloRural}, contando con más de 2 millones de colmenas registradas en 2018 \cite{SecretariaDeAgriculturaDesarrolloRural2022}.
La SADER tiene clasificadas 5 regiones apícolas en el territorio mexicano: región Norte, región Golfo, región Costa del Pacífico, región Altiplano y región Sureste o península de Yucatán; siendo esta última la de mayor producción, con un 32\% de la producción nacional, en contraste con la región Norte, que produce un 7\% \cite{SecretariaDeAgriculturaDesarrolloRural2022}.
Además de la producción de miel, las colmenas cumplen el rol de brindar el servicio de polinización. Tanto la producción bovina como la agricultura en general necesitan de este servicio. Por este motivo, aproximadamente 150\,000 colmenas están destinadas a la polinización de plantas para consumo de ganado y producción agrícola \cite{SecretariaDeAgriculturaDesarrolloRural2022}.
En cuanto al panorama estatal, Sonora se encuentra dentro de la región Norte \cite{coordinacion_general_de_ganaderia_2015}, participando con el 7\% de la producción de miel del país. Sin embargo, este no es el fin principal de la apicultura sonorense, ya que, así como en varios estados de la región Norte, el objetivo principal de las colmenas es la polinización de cultivos. Sonora contaba con 24\,000 colmenas registradas en 2008 para este fin \cite{coordinacion_general_de_ganaderia_2015}.
Otro factor que se añade al peso de la importancia de la polinización es el sector de la producción bovina, Sonora es el principal productor, con una producción bruta total de \$4\,150 millones de MXN en 2019, correspondiente a aproximadamente un 34\% de la producción nacional \cite{data_mexico_2023b}.
Esta conexión entre la apicultura, agricultura y ganadería resalta la necesidad de mantener un monitoreo constante de las colmenas apícolas y de optimizar dicho monitoreo, una de las opciones para realizar esto es mediante un sistema automático como el presentado en \cite{chen_chien_hsu_jing_lin_lin_2020} que se utilice para la medición de variables físicas relevantes de la colmena, tomando en cuenta la adición de un proyecto como el presentado en \cite{marina_lara_meza_2023} que contemple un sistema informático para el procesamiento y visualización de los datos.
El monitoreo de colmenas apícolas mediante técnicas de Internet de las cosas (IoT) ha ganado relevancia en los últimos años. Actualmente, existen diversas opciones comerciales y propuestas de diseño de sistemas de monitoreo, reflejando el interés en las ventajas que ofrece un sistema automatizado
Las opciones comerciales son consideradas como alternativas confiables, ya sea como productos independientes, como \cite{Solutionbee}, o como sistemas con versatilidad mínima pero aplicables en un amplio rango de situaciones, con \cite{OpenEnergyMonitor} como ejemplo. Sin embargo, estas opciones tienden a requerir un presupuesto más elevado y presentan limitaciones en cuanto a la personalización, ya que suelen carecer de modularidad.
En el ámbito de la investigación pública, se distinguen principalmente dos tipos de proyectos. Por un lado, proyectos diseñados específicamente para zonas particulares, como es el caso de \cite{chen_chien_hsu_jing_lin_lin_2020}. Por otro lado, proyectos que se centran en aspectos muy específicos del sistema de monitoreo, tal como lo hace el sistema de reconocimiento de audio descrito en \cite{kulyukin_mukherjee_amlathe_2018}.
